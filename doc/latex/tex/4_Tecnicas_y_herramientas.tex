\capitulo{4}{Técnicas y herramientas}
\section{Metodologías}
\subsection{Scrum}
Scrum~\cite{wiki:scrum} es un marco que se emplea para la gestión de proyectos de metodología ágil y que facilita a los equipos la tarea de gestionar y estructurar el trabajo. Para ello, a través de iteraciones (sprints) se realiza una serie de tareas en un periodo de tiempo y se van incorporando al software.
\subsection{Gitflow}
Gitflow~\cite{gitflow} se define como un flujo de trabajo para la creación de ramas en Git y llevar un control de versiones. En este flujo existen dos ramas principales, master y develop, y otras ramas que pueden ayudarnos en nuestro desarrollo como pueden ser: feature, release o  hotfix.
\section{Patrón de diseño}
\subsection{Model-View-ViewModel MVVM}
Es un patrón MVVM~\cite{mvvm}, también conocido como Model View ViewModel que se centra en separar la interfaz del usuario (View) de la parte lógica (Model). La interacción entre la parte lógica y la interfaz del usuario a través del recurso ViewModel.

Algunas de las ventajas al usar este patrón son:
\begin{itemize}
\item Fácil desarrollo ya que al poder separar la vista de la lógica varios equipos pueden trabajar simultáneamente en varios componentes.
\item Fácil testeo ya que no es necesario utilizar la vista para crear tests para el model o el viewmodel.
\item Fácil mantenimiento ya que al realizar la separación de los componentes se crea un código simple y limpio.
\end{itemize}

Vamos a describir a continuación cada uno de los componentes que forman el MVVM:
\begin{itemize}
	\item Model: es el componente donde se encapsulan los datos de nuestra aplicación. En ella se pueden encontrar validación y lógica de negocio.
	\item View: nos muestra el diseño y la apariencia de nuestra aplicación. En ella se verán los datos pero sin contener nada de la lógica de negocio.
	\item ViewModel: es el componente que enlaza los datos o cambios de estado que puede tener nuestra aplicación.
\end{itemize}

\imagen{mvvm-pattern}{Arquitectura limpia}{.9}

\section{Repositorio}
Entre las herramientas consideradas GitHub~\cite{github}, Bitbucket~\cite{bitbucket}, GitLab~\cite{wiki:gitlab} se decide utilizar GitHub, porque nos permite alojar proyectos gratuitamente. Además podemos crear documentación a través de wikis, crear tareas, sprints.
Al crear el respositorio en GitHub podemos usar el gestor de proyecto que tiene a través de sus paneles y saber el estado en el que está cada issue (tarea).
Otra de las ventajas que encontramos con GitHub es que está integrado en multiples servicios de integración continua.
\section{Control de versiones}
Al haber escogido como repositorio GitHub para alojar nuestro proyecto, utilizamos como control de versiones Git~\cite{wiki:git} que es un sistema distribuido que nos permite realizar nuestro trabajo sin conexión, es ligero y rápido.
El crear ramas y hacer los merges es muy rápido y pocas veces se obtienen conflictos.
El historial es muy detallado.
\section{Gestión del proyecto}
Como gestor de proyectos utilizamos el que viene integrado en GitHub. Con esta herramienta podemos gestionar nuestro proyecto y ver en que estado está cada tarea, incluso podemos darle prioridad.
\section{Entorno de desarrollo integrado (IDE)}
\subsection{Flutter}
Flutter~\cite{flutter} es un SDK (Kit de Desarrollo de Software) desarrollado por Google que nos permite crear aplicaciones para varios dispositivos. Hemos elegido esta herramienta porque nos permite crear aplicaciones multiplataforma.
Entre alguna de las ventajas destacamos:
\begin{itemize}
\item \textbf{Compila en nativo} tanto para Android como iOS.
\item Mediante el uso de widgets, se pueden \textbf{crear interfaces gráficas flexibles}.
\item \textbf{Rapidez} en el desarrollo.
\end{itemize}
\subsection{Dart}
Dart~\cite{dart} es un \textbf{lenguaje de programación moderno}, desarrollado por Google, donde se mezcla la programación orientada a objetos POO con los lenguajes de programación basados en scripts.
\subsection{FlutterFlow}
FlutterFlow~\cite{flutterflow} es una herramienta que nos permite crear aplicaciones insertando los componentes que tiene la herramienta, arrastrando y soltando. Es posible añadir lógica durante la creación de las páginas.
\subsection{LaTeX}
\subsection{Documentación}
\section{Integración y entregas continuas (CI/CD)}
\subsection{Codemagic}
Codemagic~\cite{codemagic} es producto basado en la nube que permite realizar integración y entregas continuas. Te permite crear workflow versátiles y activar la compilación,  análisis de código y pruebas de manera automática.
\section{Herramientas contrucción del proyecto}
\subsection{Librerías}



