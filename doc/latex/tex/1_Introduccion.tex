\capitulo{1}{Introducción}

El proyecto nace para dar solución y mejorar la calidad de vida de personas que tengan \textbf{Esclerosis Múltiple}~\cite{wiki:em,medlineplus:em} (\textbf{EM}). Para ello se va a desarrollar el Frontend inicialmente para dispositivo móvil Android. 

En un primer momento aunque hemos tenido antecedentes familiares con este tipo de enfermedad buscamos más información, resumiendo podemos decir que la EM es una enfermedad neurológica crónica cuya naturaleza es inflamatoria y autoinmunitaria. Se caracteriza por la afectar el cerebro y la médula espinal y hace más lentos o bloquea la conexión entre el cerebro y el cuerpo.
Algunos síntomas de EM pueden ser:
\begin{itemize}
\item Alteraciones en la visión.
\item Debilidad a nivel muscular.
\item Problemas con la coordinación y el equilibrio.
\item Apacición de lesiones cerebrales detectadas en la resonancia magnética.
\end{itemize}
Podemos clasificar esta enfermedad entre:
\begin{itemize}
\item \textbf{Recurrente-remitente (EMRR)} caracterizada por aparición periódica de ataques o brotes.
\item \textbf{Secundaria progresiva (EMSP)} se caracteriza por el empeoramiento gradual y progresiva de la discapacidad.
\item \textbf{Primaria progresiva (EMPP)} este tipo comienza de manera inofensiva y empeora con el tiempo.
\item \textbf{Progresiva-recurrente (EMPR)} este es el más agresivo ya que las apariciones de los síntomas son severas y periódicas.
\end{itemize}

Para la creación de la aplicación se tendrá en cuenta aquellos aspectos que pueden aportar mejoras en la calidad de vida del usuario como puede ser:
\begin{itemize}
\item Recordatorios de la medicación o el tratamiento
\item Recordatorios de las citas
\end{itemize}



