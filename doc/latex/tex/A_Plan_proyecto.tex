\apendice{Plan de Proyecto Software}

\section{Introducción}
La finalidad de este proyecto es ofrecer la oportunidad para mejorar la calidad de vida de las personas 
que han sido diagnosticadas con algún tipo de Esclerósis Múltiple (EM). Para ello se va a desarrollar una 
aplicación frontend que de manera amigable dará soporte a aquellas personas que hayan decido utilizarla.

En este apartado se valorarán y analizarán las la planificación temporal así como el estudio de viabilidad.

\section{Planificación temporal}
Para la planificación temporal se tendrán en cuenta el tiempo que conlleva la curva de aprendizaje para 
poner en marcha el proyecto, así como las distintas funcionalidades que se van implementando según las 
necesidades aportadas por el cliente.
A través del tiempo se realizan sprints para poder desarrollar y añadir las funcionalidades correspondientes al
proyecto.

\begin{itemize}
\tightlist
\item
Se aplicó una estrategia de desarrollo a partir de (\emph{sprints}) mediante iteraciones incrementales.
\item
Se realizan \emph{sprints} que conllevan poco tiempo para la creación de funcionalidades.
\item
Al finalizar el \emph{sprint} correspondiente se incorpora a la release que posteriormente formará parte del producto.
\item
Se realizaban reuniones con el cliente para revisar el producto. En estas revisiones se aportan mejoras/modificaciones.
\item
Tras la obtención de requisitos de cliente, se planifican las nuevas tareas y que deciden en que \emph{sprint} estarán asignadas.
\item
Cada tarea incluida se estima en tiempo en un tablero \emph{scrum}.
\item
Cada tarea estimada se priorizan dentro del tablero \emph{scrum}.
\end{itemize}

\subsection{Sprint 0 (19/10/2023)}
Se realiza reunión con el tutor Pedro Renedo Fernández para asignación del proyecto para realizar la parte frontend de una aplicación
para dar soporte a usuarios diagnoticados con Esclerósis Múltiples (EM).

En ella se comenta una idea inicial sobre que herramientas se van a emplear y la selección del framework para crear el proyecto.

\subsection{Sprint 1 (27/10/2023)}
Se realiza una primera toma de contacto con el cliente con los requisitos iniciales. En ella nos transmite que está muy ilusionado en 
que queramos aceptar el reto y poder dar a luz un aplicación que podrá facilitar la vida de mucha gente.

Para los objetivos iniciales nos informa que está interesado en crear una aplicación móvil para personas que hayan sido diagnosticadas
con EM y así poder aportar una solución en su día a día.

Le indicamos que tenemos intención en realizar la aplicación en un lenguaje que podría utilizarse en distintos dispositivos así como podría 
ser usada en distintos Sistemas Operativos. 

Inicialmente nos indica que una versión inicial estaría disponible para dispositivos con sistema operativo Android.

\subsection{Sprint 2 (28/10/2023 al 04/02/2024)}
Se realiza una primera toma de contacto con el framework Flutter después de analizar que ventajas e inconvenientes podríamos encontrarnos
para realizar la aplicación móvil solicitada por el cliente.

Tras tomar la decisión se empieza una formación inicial en el lenguaje de programación Dart para entender su funcionamiento y poder realizar
el desarrollo del proyecto.

\subsection{Sprint 3 (05/10/2023 al 11/02/2024)}
Se realiza la creación del proyecto con el framework Flutter y con el lenguaje de programación Dart. Además, se añaden las configuraciones iniciales
y se crea la página inicial.

\subsection{Sprint 4 (12/02/2024 al 18/02/2024)}
Se añaden las páginas de de Login y Registro al proyecto.

\subsection{Sprint 5 (19/02/2024 al 25/02/2024)}
Crear página detalle del usuario y home.

\subsection{Sprint 6 (26/02/2024 al 10/03/2024)}
Crear aplicación responsive para varios dispositivos (web, tablet, móvil)
Corregir bug de idioma
Enviar correo al cliente para resolver dudas a cerca de los requisitos del usuario.

\section{Estudio de viabilidad}

\subsection{Viabilidad económica}

\subsection{Viabilidad legal}


